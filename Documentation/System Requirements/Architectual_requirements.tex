\documentclass[11pt, a4paper]{article}

% --- PREAMBLE ---
\usepackage[margin=1in, landscape]{geometry}
\usepackage{longtable}
\usepackage{booktabs}
\usepackage{array}
\usepackage{enumitem}
\usepackage[table]{xcolor}
\usepackage{hyperref}

% Define colors and column types
\definecolor{tablehead}{gray}{0.92}
\newcolumntype{L}[1]{>{\raggedright\arraybackslash}p{#1}}

% --- DOCUMENT START ---
\begin{document}

\begin{longtable}{L{0.14\textwidth} L{0.2\textwidth} L{0.2\textwidth} L{0.36\textwidth}}

% --- TABLE CAPTION ---
\caption{Mavito Project: Architecture Mapping from Requirements to Implementation} \\
\toprule 

% --- TABLE HEADER ---
\rowcolor{tablehead}
\textbf{Requirement} & \textbf{Architectural Strategies Used} & \textbf{Architectural Pattern Used} & \textbf{Our Specific Implementation (Mavito Project)} \\
\midrule
\endfirsthead

% This header will repeat on all subsequent pages
\caption{Mavito Project: Architecture Mapping (continued)} \\
\toprule
\rowcolor{tablehead}
\textbf{Requirement} & \textbf{Architectural Strategies Used} & \textbf{Architectural Pattern Used} & \textbf{Our Specific Implementation (Mavito Project)} \\
\midrule
\endhead

% --- TABLE CONTENT ---

% Row 1: Scalability
\textbf{Scalability} &
\begin{itemize}[nosep, leftmargin=*]
    \item Horizontal scale-out
\end{itemize} &
Microservices &
\begin{itemize}[nosep, leftmargin=*]
    \item \textbf{Backend:} Services are packaged in \texttt{Docker} containers and deployed to \textbf{Google Cloud Run} for request-based auto-scaling.
    \item \textbf{Frontend:} Hosted as a static site (e.g., on GitHub Pages or Firebase Hosting), which scales globally for content delivery.
\end{itemize} \\
\midrule

% Row 2: Performance
\textbf{Performance} &
\begin{itemize}[nosep, leftmargin=*]
    \item Async APIs
    \item Database Indexing
\end{itemize} &
Microservices with Asynchronous APIs &
\begin{itemize}[nosep, leftmargin=*]
    \item \textbf{Backend:} Built with \textbf{FastAPI} using an \texttt{async/await} model for non-blocking I/O, ensuring low-latency responses.
    \item \textbf{Database:} The \texttt{search-service} now queries an indexed \textbf{PostgreSQL} database, allowing for highly efficient filtering and sorting of millions of records.
\end{itemize} \\
\midrule

% Row 3: Availability
\textbf{Availability} &
\begin{itemize}[nosep, leftmargin=*]
    \item Replication
\end{itemize} &
Leader-Follower Replication &
\begin{itemize}[nosep, leftmargin=*]
    \item \textbf{Database:} \textbf{Google Cloud SQL} for PostgreSQL can be configured for High Availability (HA) to manage replication and failover automatically.
    \item \textbf{Services:} \textbf{Google Cloud Run} is a managed service that ensures services are reliable and restarted on failure.
\end{itemize} \\
\midrule

% Row 4: Usability & Data Integrity
\textbf{Usability} \& \textbf{Data Integrity} &
\begin{itemize}[nosep, leftmargin=*]
    \item Real-time UI
    \item Persistent State
\end{itemize} &
Component-Based UI & % <-- CORRECTED PATTERN
\begin{itemize}[nosep, leftmargin=*]
    \item \textbf{Frontend:} A \textbf{React} and \texttt{TypeScript} application follows a component-based pattern to separate UI concerns.
    \item \textbf{Backend Database:} The `search-service` and `vote-service` now use \textbf{PostgreSQL} as the single source of truth for all terminology and vote data, ensuring data integrity.
\end{itemize} \\
\midrule

% Row 5: Security
\textbf{Security} &
\begin{itemize}[nosep, leftmargin=*]
    \item TLS \& tokens
    \item Secure Cloud Storage
\end{itemize} &
API Gateway &
\begin{itemize}[nosep, leftmargin=*]
    \item \textbf{Gateway Service:} \textbf{Google API Gateway} serves as the single entry point, handling TLS termination (enforcing HTTPS).
    \item \textbf{Authentication:} \textbf{JWT tokens} are managed by the `auth-service` to secure protected API endpoints like the vote submission endpoint.
    \item \textbf{File Uploads:} User-submitted documents are securely uploaded directly to a private \textbf{Google Cloud Storage} bucket using temporary signed URLs.
\end{itemize} \\
\midrule

% Row 6: Offline & Portability
\textbf{Offline Accessibility} \& \textbf{Portability} &
\begin{itemize}[nosep, leftmargin=*]
    \item Service workers and caching
    \item Background Sync
\end{itemize} &
Progressive Web App (PWA) &
\begin{itemize}[nosep, leftmargin=*]
    \item \textbf{Frontend:} Built as a PWA with a Service Worker to support offline access. API POST requests (like votes) are queued using \textbf{Workbox Background Sync} and sent automatically upon reconnection.
    \item \textbf{Backend:} The backend is fully containerized with \textbf{Docker}, ensuring it is portable and can run consistently in any environment.
\end{itemize} \\
\midrule

% Row 7: Maintainability & DevOps
\textbf{Maintainability} \& \textbf{Deployment} &
\begin{itemize}[nosep, leftmargin=*]
    \item Modular design
    \item CI/CD Automation
    \item Database Migrations
\end{itemize} &
Automated Testing \& Deployment Pipeline &
\begin{itemize}[nosep, leftmargin=*]
    \item \textbf{Code Quality:} \textbf{Husky pre-commit hooks} enforce \texttt{Ruff}, \texttt{Black}, and \texttt{Mypy} checks locally inside Docker containers.
    \item \textbf{CI/CD:} A \textbf{GitHub Actions} workflow automates testing and quality checks for both frontend and backend.
    \item \textbf{Database Schema:} A dedicated `alembic-service` is used to manage and apply database migrations, ensuring a consistent schema across all environments.
\end{itemize} \\
\bottomrule

\end{longtable}

\end{document}
