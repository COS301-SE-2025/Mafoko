\documentclass[12pt]{article}
\usepackage[a4paper, margin=1in]{geometry}
\usepackage{titlesec}
\usepackage{hyperref}
\usepackage{graphicx}
\usepackage{enumitem}
\usepackage{longtable}
\usepackage{array}
\usepackage{ragged2e}
\usepackage{float}
\usepackage{listings}
\usepackage{xcolor}

\titleformat{\section}[block]{\large\bfseries}{\thesection.}{0.5em}{}
\titleformat{\subsection}[block]{\normalsize\bfseries}{\thesubsection.}{0.5em}{}

% Code listing style
\definecolor{codegreen}{rgb}{0,0.6,0}
\definecolor{codegray}{rgb}{0.5,0.5,0.5}
\definecolor{codepurple}{rgb}{0.58,0,0.82}
\definecolor{backcolour}{rgb}{0.95,0.95,0.92}

\lstdefinestyle{mystyle}{
    backgroundcolor=\color{backcolour},   
    commentstyle=\color{codegreen},
    keywordstyle=\color{codepurple},
    numberstyle=\tiny\color{codegray},
    stringstyle=\color{codegreen},
    basicstyle=\ttfamily\footnotesize,
    breakatwhitespace=false,         
    breaklines=true,                 
    captionpos=b,                    
    keepspaces=true,                 
    numbers=left,                    
    numbersep=5pt,                  
    showspaces=false,                
    showstringspaces=false,
    showtabs=false,                  
    tabsize=2
}

\lstset{style=mystyle}

\title{Testing Policy Documentation\\\large\textbf{Marito Multilingual Terminology PWA}}
\author{Team Name: Velox}
\date{September 7, 2025}

\begin{document}

\maketitle
\tableofcontents
\newpage

\section{Introduction}
This document outlines the testing policy and procedures for the Marito project. It describes our approach to testing, tools used, and the processes we follow to ensure high-quality software delivery.

\section{Testing Framework}

\subsection{Testing Levels}
\begin{itemize}
    \item \textbf{Unit Testing}: Testing individual components and functions
    \item \textbf{Integration Testing}: Testing interactions between components
    \item \textbf{End-to-End Testing}: Testing complete user workflows
    \item \textbf{Performance Testing}: Testing system performance under load
\end{itemize}

\section{Testing Tools}

\subsection{Frontend Testing}
\begin{itemize}
    \item \textbf{Vitest}: Primary testing framework for React components
    \item \textbf{React Testing Library}: For testing React components in a user-centric way
    \item \textbf{Cypress}: For end-to-end testing
\end{itemize}

\subsection{Backend Testing}
\begin{itemize}
    \item \textbf{Pytest}: Primary testing framework for Python services
    \item \textbf{Coverage.py}: For measuring code coverage
    \item \textbf{Locust}: For load testing
\end{itemize}

\section{Continuous Integration}

\subsection{GitHub Actions}
We use GitHub Actions as our CI/CD platform instead of Travis CI for the following reasons:
\begin{itemize}
    \item Native integration with GitHub repositories
    \item More generous free tier for open-source projects
    \item Better support for matrix builds and parallel testing
    \item Faster build times and more concurrent jobs
    \item Built-in secret management
\end{itemize}

\section{Testing Procedure}

\subsection{Local Testing}
\begin{enumerate}
    \item Developers must write tests for new features
    \item All tests must pass locally before committing
    \item Run \texttt{npm test} for frontend tests
    \item Run \texttt{pytest} for backend tests
\end{enumerate}

\subsection{Automated Testing}
\begin{enumerate}
    \item Tests run automatically on pull requests
    \item Code coverage reports are generated
    \item Performance tests run nightly
    \item Security scanning is performed on dependencies
\end{enumerate}

\section{Test Repository Structure}

\subsection{Frontend Tests}
Tests are located in the \texttt{frontend/Tests} directory:
\begin{lstlisting}[frame=single]
frontend/
  Tests/
    components/
    pages/
    utils/
    e2e/
\end{lstlisting}

\subsection{Backend Tests}
Each service has its own tests directory:
\begin{lstlisting}[frame=single]
backend/
  service-name/
    tests/
      unit/
      integration/
      e2e/
\end{lstlisting}

\section{Testing Standards}

\subsection{Coverage Requirements}
\begin{itemize}
    \item Minimum 80\% code coverage for new features
    \item Critical paths must have 100\% coverage
    \item Integration tests for all API endpoints
    \item E2E tests for main user workflows
\end{itemize}

\subsection{Testing Best Practices}
\begin{itemize}
    \item Write tests before implementation (TDD)
    \item Keep tests focused and atomic
    \item Use meaningful test descriptions
    \item Mock external dependencies
    \item Follow AAA pattern (Arrange-Act-Assert)
\end{itemize}

\section{Git Repository}
All test cases and reports can be found in our GitHub repository:
\begin{itemize}
    \item Repository: \url{https://github.com/COS301-SE-2025/Marito}
    \item Test Documentation: \texttt{/Documentation/Testing Policy/}
    \item CI/CD Workflows: \texttt{/.github/workflows/}
\end{itemize}

\end{document}